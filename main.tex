\documentclass{article}
\usepackage[T1]{fontenc}
\usepackage[utf8]{inputenc}
\usepackage{amsmath, amssymb, amsthm}

\usepackage{palatino}
\usepackage{enumitem}
\usepackage{makecell}

\usepackage[colorlinks]{hyperref}

\usepackage{tikz-cd}

\newtheorem{prop}{Proposition}
\newtheorem{thm}{Theorem}
\theoremstyle{definition}
\newtheorem{defn}{Definition}
\newtheorem{examples}{Examples}

\newcommand{\C}{\mathsf{C}}
\newcommand{\D}{\mathsf{D}}
\newcommand{\M}{\mathbf{M}}
\newcommand{\ZZ}{\mathbb{Z}}
\DeclareMathOperator{\id}{id}
\newcommand{\Alg}{\mathsf{Alg}}
\newcommand{\Free}{\mathsf{Free}}
\DeclareMathOperator{\Hom}{Hom}

\title{Compact Hausdorff Spaces are Algebraic Objects}
\author{Ben Spitz}
\date{March 2023}

\begin{document}

\maketitle

A monad is a categorical gadget that encodes an ``algebraic theory''. For example, there is a monad which encodes the theory of groups, and a monad which encodes the theory of rings. It is useful to be able to identify when a theory is encoded by a monad, because theories encoded by a monad enjoy many useful categorical properties.

For example, isomorphisms in such theories are always detected on underlying objects -- e.g. a group or ring homomorphism is an isomorphism if and only if it is bijective.

In this document, we'll learn the basic theory of monads, and come to an interesting conclusion: the theory of compact hausdorff spaces is encoded by a monad!

\section{Motivating Example}

Let $X$ be a set. The \emph{free abelian group} on $X$ consists of the formal $\ZZ$-linear combinations of elements of $X$, which comes equipped with a ``formal sum'' operation (we can add any two linear combinations to produce a new linear combination). Let's denote this set of formal $\ZZ$-linear combinations by $T(X)$ -- to be clear, $T(X)$ is just the set; we are not remembering the formal sum operation!

Now, this formal sum operation can be made a bit more general: instead of just adding two linear combinations, we can just as well interpret any linear combination of linear combinations! This gives a function
\[\mu_X : T(T(X)) \to T(X).\]

For example, if $X = \{a,b\}$, then $a+b$ and $2a-b$ are elements of $T(X)$, and $2(a+b)-3(2a-b)$ is an element of $T(T(X))$. In this example, $$\mu_X(2(a+b)-3(2a-b)) = -4a+5b.$$

It is not just free abelian groups in which we can talk about $\ZZ$-linear combinations. In fact, for \emph{any} abelian group $(A,*)$, we can use the operation $*$ to evaluate formal $\ZZ$-linear combinations of elements of $A$. In other words, we get a function $\varepsilon_A : T(A) \to A$ which encodes the group operation of $(A,*)$, just as $\mu_X$ encoded the formal sum operation on $T(X)$.

We can also try to go the other way -- for any set $Y$ and any function $e : T(Y) \to Y$, we may try to produce an abelian group operation $*$ on $Y$ as follows:
\[a * b = e(a + b).\]
Put simply: we view $e$ as a way to evaluate formal linear combinations, which should tell us how to add elements of $Y$. With this in mind, we should impose a condition on $e$: we will assume $e(y) = y$ for all $y \in Y$. In other words, we assume that $e$ evaluates all ``singleton linear combinations'' in the canonical way.

Now $*$ certainly defines a binary operation on $Y$, but it may or may not be an abelian group operation. We would need to check the abelian group axioms, which are abbreviated as follows:
\begin{enumerate}
    \item Associativity
    \item Commutativity
    \item Identity
    \item Inverses
\end{enumerate}
This is quite a lot to check! Writing all of this out in terms of $e$ and formal $\ZZ$-linear combinations would be a pain. However, there is a nice characterization of when $*$ is an abelian group operation: $e$ simply needs to respect linear combinations of linear combinations! Precisely:

\begin{prop}
$*$ is an abelian group operation on $Y$ if and only if the following diagram commutes:
\[\begin{tikzcd}
T(T(Y)) \rar["\mu_Y"] \dar["T(e)"] & T(Y) \dar["e"] \\
T(Y) \rar["e"] & Y
\end{tikzcd}\]
where $T(e)$ is the function $T(T(Y)) \to T(Y)$ defined by sending a formal linear combination $a_1 f_1 + \dots + a_n f_n$ (where $a_1, \dots, a_n \in \ZZ$ and $f_1, \dots, f_n \in T(Y)$) to $a_1 e(f_1) + \dots + a_n e(f_n)$.
\end{prop}

Let's unpack the diagram:

The top-right composition $e \circ \mu_Y$ takes a formal linear combination of formal linear combinations, collapses it to a formal linear combination using $\mu_Y$, and then evaluates it to an element of $Y$ using $e$.

The left-bottom composition $e \circ T(e)$ takes a formal linear combination of formal linear combinations, uses $e$ to evaluate the inner linear combinations, then uses $e$ again to evaluate the outer linear combination.

Having these two compositions be equal is (vaguely) saying that $e$ interprets formal linear combinations in a way that is compatible with the way that formal linear combinations need to interact with one another.

We will not prove this proposition in full, but let's see how one part of this works -- we'll prove that if the diagram commutes, then $*$ has an identity element.

\begin{proof}
Consider the element $0 \in T(Y)$ (the empty linear combination). We claim that $e(0)$ is the identity element of $*$. So, let $y \in Y$ be arbitrary. Then
$$e(0) * y = e(e(0) + y) = e(e(0) + e(y)) = e(T(e)(0+y)) = e(\mu_Y(0+y)) = e(y) = y$$
and
$$y * e(0) = e(y + e(0)) = e(e(y) + e(0)) = e(T(e)(y+0)) = e(\mu_Y(y+0)) = e(y) = y.$$
Since $y$ is arbitrary, this completes the proof.
\end{proof}

The proofs of the other axioms are mostly similar.

Note: the function $\mu_X : T(T(X)) \to T(X)$ is exactly the function $e$ which encodes the abelian group structure of the free abelian group on $X$! You can check that indeed $\mu_X(f) = f$ for all $f \in T(X)$, and the desired diagram commutes.

In general, a \emph{monad} consists of a ``free construction'' (like $X \mapsto T(X)$) with an associated ``evaluation map'' (like $\mu_X$) and a way to view elements of $X$ as elements of $T(X)$ (in this example: singleton linear combinations).

Then an \emph{algebra} for the monad (in this example: an abelian group) is given by a map $T(X) \to X$ which plays nicely with all of this data, as described above.

And, in order for $X \mapsto T(X)$ to count as a ``free construction'', the map $\mu_X : T(T(X)) \to T(X)$ should exhibit $T(X)$ as an algebra for the monad (in this example: $\mu_X$ encodes the group operation on $T(X)$)!

In the next section, we'll make precise definitions out of this.

\section{Definitions}

Let $\C$ be a category.

\begin{defn}\label{def:monad}
A \emph{monad} on $\C$ is a triple $(M, \mu, \eta)$ where:
\begin{enumerate}[label = (\roman*.)]
    \item\label{def:monad:data:functor} $M : \C \to \C$ is a functor;
    \item\label{def:monad:data:multiplication} $\mu : M \circ M \to M$ is a natural transformation;
    \item\label{def:monad:data:unit} $\eta : \id_\C \to M$ is a natural transformation;
    \item\label{def:monad:condition:id} the following diagram commutes:
    % https://q.uiver.app/?q=WzAsNixbMCwwLCJNIFxcY2lyYyBcXGlkX1xcQyJdLFsxLDAsIk0iXSxbMiwwLCJcXGlkX1xcQyBcXGNpcmMgTSJdLFswLDEsIk0gXFxjaXJjIE0iXSxbMiwxLCJNIFxcY2lyYyBNIl0sWzEsMSwiTSJdLFswLDEsIj0iLDEseyJzdHlsZSI6eyJib2R5Ijp7Im5hbWUiOiJub25lIn0sImhlYWQiOnsibmFtZSI6Im5vbmUifX19XSxbMSwyLCI9IiwxLHsic3R5bGUiOnsiYm9keSI6eyJuYW1lIjoibm9uZSJ9LCJoZWFkIjp7Im5hbWUiOiJub25lIn19fV0sWzAsMywiTSBcXGNpcmMgXFxldGEiLDJdLFsyLDQsIlxcZXRhIFxcY2lyYyBNIl0sWzMsNSwiXFxtdSIsMl0sWzQsNSwiXFxtdSJdLFsxLDUsIlxcaWRfTSJdXQ==
    \[\begin{tikzcd}[ampersand replacement=\&]
    	{M \circ \id_\C} \& M \& {\id_\C \circ M} \\
    	{M \circ M} \& M \& {M \circ M}
    	\arrow[equals, from=1-1, to=1-2]
    	\arrow[equals, from=1-2, to=1-3]
    	\arrow["{M \circ \eta}"', from=1-1, to=2-1]
    	\arrow["{\eta \circ M}", from=1-3, to=2-3]
    	\arrow["\mu"', from=2-1, to=2-2]
    	\arrow["\mu", from=2-3, to=2-2]
    	\arrow["{\id_M}", from=1-2, to=2-2]
    \end{tikzcd}\]
    \item\label{def:monad:condition:assoc} the following diagram commutes:
    % https://q.uiver.app/?q=WzAsNCxbMCwwLCJNIFxcY2lyYyBNIFxcY2lyYyBNIl0sWzEsMCwiTSBcXGNpcmMgTSJdLFswLDEsIk0gXFxjaXJjIE0iXSxbMSwxLCJNIl0sWzAsMSwiTSBcXGNpcmMgXFxtdSJdLFswLDIsIlxcbXUgXFxjaXJjIE0iLDJdLFsyLDMsIlxcbXUiLDJdLFsxLDMsIlxcbXUiXV0=
    \[\begin{tikzcd}[ampersand replacement=\&]
    	{M \circ M \circ M} \& {M \circ M} \\
    	{M \circ M} \& M
    	\arrow["{M \circ \mu}", from=1-1, to=1-2]
    	\arrow["{\mu \circ M}"', from=1-1, to=2-1]
    	\arrow["\mu"', from=2-1, to=2-2]
    	\arrow["\mu", from=1-2, to=2-2]
    \end{tikzcd}\]
\end{enumerate}
Equivalently, if you're the kind of person who prefers this: a monad is a monoid object in the strict monoidal category $(\operatorname{End}(\C), \circ)$.
\end{defn}

\begin{examples}$\ $
\begin{enumerate}[label = \textbf{\Alph*.}]
    \item To connect with motivating example in the previous section: we used $\C = \mathsf{Set}$, and the functor under consideration was called $T$. The natural transformation $\mu$ was exactly the map $\mu_X : T(T(X)) \to T(X)$ that we described, and the natural transformation $\eta$ was the function $X \to T(X)$ which sends each element of $X$ to its singleton linear combination.
    \item Let $M : \mathsf{Ab} \to \mathsf{Ab}$ be the functor $\mathbb{R} \otimes {-}$. There is a natural map $\eta_X : X \to \mathbb{R} \otimes X$ given by $x \mapsto 1 \otimes x$, and a natural map $\mu_X : \mathbb{R} \otimes (\mathbb{R} \otimes X) \to \mathbb{R} \otimes X$ given by linearly extending $r \otimes (s \otimes x) \mapsto rs \otimes x$. The triple $(M, \mu, \eta)$ is a monad on $\mathsf{Ab}$.
    \item The above example generalizes: simply replace $\mathbb{R}$ with any ring whatsoever.
    \item Let $P : \mathsf{Set} \to \mathsf{Set}$ be the (covariant) power set functor, which sends each set $X$ to its power set $P(X)$, and sends a morphism $f : X \to Y$ to the morphism $P(f) : P(X) \to P(Y)$ defined by $P(f)(A) = \{f(a) : a \in A\}$. There is natural map $\eta_X : X \to P(X)$ given by $x \mapsto \{x\}$, and a natural map $\mu_X : P(P(X)) \to P(X)$ given by $S \mapsto \bigcup S$. The triple $(P, \mu, \eta)$ is a monad on $\mathsf{Set}$.
    \item Let $L : \mathsf{Set} \to \mathsf{Set}$ be the functor which sends each set $X$ to the set $L(X) := \coprod_{n \geq 0} X^n$ of finite strings in $X$. There is a natural map $\eta_X : X \to L(X)$ which sends each element of $X$ to the corresponding $1$-letter string, and a natural map $\mu_X : L(L(X)) \to X$ which takes a string of strings and concatenates it. The triple $(L, \mu, \eta)$ is a monad on $\mathsf{Set}$.
    \item Let $G : \mathsf{Set} \to \mathsf{Set}$ be the functor which sends each set $X$ to the set $G(X)$ of finitely-supported probability measures on $X$ (with the discrete $\sigma$-algebra), and sends a function $f : X \to Y$ to the pushforward function $G(f) : G(X) \to G(Y)$ given by $G(f)(\nu) = f_* \nu$. There is a natural map $\eta_X : X \to G(X)$ given by $x \mapsto \delta_x$, and a natural map $\mu_X : G(G(X)) \to G(X)$ given by $\mu_X(N)(E) = \int_{G(X)} \nu(E) \; \mathrm{d}N(\nu)$. The triple $(G, \mu, \eta)$ is a monad on $\mathsf{Set}$.
    \item Let $B : \mathsf{Top} \to \mathsf{Top}$ be the functor which sends a space $X$ to the space $B(X) := X \amalg \{*\}$ and which sends a morphism $f : X \to Y$ to the morphism $X \amalg \{*\} \to Y \amalg \{*\}$ given by $x \mapsto f(x)$ for $x \in X$ and $* \mapsto *$. There is a natural inclusion $\eta_X : X \to B(X)$, and there is a natural map $B(B(X)) = (X \amalg \{*\}) \amalg \{*\} \to B(X)$ given by sending both copies of $*$ to the same point $* \in B(X)$. The triple $(B,\mu,\eta)$ is a monad on $\mathsf{Top}$.
    \item The above example generalizes: simply replace $\mathsf{Top}$ with any category which has binary coproducts and a terminal object.
    \item Let $X$ be a topological space, and let $\mathcal{P}(X)$ denote the power set of $X$, viewed as a poset category. Let $\operatorname{cl} : \mathcal{P}(X) \to \mathcal{P}(X)$ be the functor given by $\operatorname{cl}(A) = \overline{A}$. There is an inclusion $\eta_A : A \to \overline{A}$ for all $A \subseteq X$, and an inclusion (actually an equality) $\mu_A : \overline{\overline{A}} \to \overline{A}$ for all $A \subseteq X$. The triple $(\operatorname{cl}, \eta, \mu)$ is a monad on $\mathcal{P}(X)$.
    \item Let $P : \mathsf{Quiv} \to \mathsf{Quiv}$ be the functor which sends a quiver $Q$ to its path quiver $P(Q)$ (and sends a quiver morphism to its natural action on paths). There is a natural map $\eta_Q : Q \to P(Q)$ which sends each vertex of $Q$ to the empty path starting at $Q$ and edge of $Q$ to the singleton path consisting of that edge. There is also a natural map $\mu_Q : P(P(Q)) \to P(Q)$ which sends a path of paths to its concatenation. The triple $(P,\mu,\eta)$ is a monad on $\mathsf{Quiv}$.
\end{enumerate}
\end{examples}

\begin{defn}
Let $\M = (M,\mu,\eta)$ be a monad on $\C$. An \emph{algebra} over $\M$ (aka an $\M$-algebra) is a pair $(X,e)$, where
\begin{enumerate}[label = (\roman*.)]
    \item $X$ is an object of $\C$;
    \item $e : M(X) \to X$ is a morphism in $\C$;
    \item the following diagram commutes:
    % https://q.uiver.app/?q=WzAsMyxbMCwwLCJYIl0sWzEsMCwiTShYKSJdLFsxLDEsIlgiXSxbMCwxLCJcXGV0YV9YIl0sWzEsMiwiZSJdLFswLDIsIlxcaWRfWCIsMl1d
    \[\begin{tikzcd}[ampersand replacement=\&]
    	X \& {M(X)} \\
    	\& X
    	\arrow["{\eta_X}", from=1-1, to=1-2]
    	\arrow["e", from=1-2, to=2-2]
    	\arrow["{\id_X}"', from=1-1, to=2-2]
    \end{tikzcd}\]
    \item the following diagram commutes:
    % https://q.uiver.app/?q=WzAsNCxbMCwwLCJNKE0oWCkpIl0sWzEsMCwiTShYKSJdLFsxLDEsIlgiXSxbMCwxLCJNKFgpIl0sWzAsMSwiXFxtdV9YIl0sWzEsMiwiZSJdLFswLDMsIk0oZSkiLDJdLFszLDIsImUiLDJdXQ==
    \[\begin{tikzcd}[ampersand replacement=\&]
    	{M(M(X))} \& {M(X)} \\
    	{M(X)} \& X
    	\arrow["{\mu_X}", from=1-1, to=1-2]
    	\arrow["e", from=1-2, to=2-2]
    	\arrow["{M(e)}"', from=1-1, to=2-1]
    	\arrow["e"', from=2-1, to=2-2]
    \end{tikzcd}\]
\end{enumerate}
\end{defn}

\begin{examples}\label{examples:algebras}$\ $
\begin{enumerate}[label = \textbf{\Alph*.}]
    \item Algebras for the motivating example monad $T$ on $\mathsf{Set}$ from the previous section are just abelian groups.
    \item Algebras for the monad $\mathbb{R} \otimes {-}$ on $\mathsf{Ab}$ are just real vector spaces.
    \item Algebras for the monad $R \otimes {-}$ on $\mathsf{Ab}$ (where $R$ is any ring) are just left $R$-modules.
    \item Algebras for the power-set monad $P$ on $\mathsf{Set}$ are just complete semilattices.
    \item Algebras for the finite-string monad $L$ on $\mathsf{Set}$ are just monoids.
    \item Algebras for the finite-probability-measure monad $G$ on $\mathsf{Set}$ are just abstract convex spaces.
    \item Algebras for the free-basepoint monad $B$ on $\mathsf{Top}$ are just pointed topological spaces.
    \item Algebras for the free-basepoint monad $B$ on an arbitrary category $\C$ with finite coproducts and a terminal object are just pointed objects in $\C$.
    \item Algebras for the closure monad $\operatorname{cl}$ on the power set of a topological space $X$ are just closed subsets of $X$.
    \item Algebras for the path monad on $\mathsf{Quiv}$ are just small categories.
\end{enumerate}
\end{examples}

We now come to our first proposition about monads, which we saw a special case of earlier:

\begin{prop}
Let $\M = (M, \mu, \eta)$ be a monad on a category $\C$. For each object $X$ of $\C$, $(M(X), \mu_X)$ is an $\M$-algebra, called the \emph{free $\M$-algebra on $X$}.
\end{prop}

\begin{proof}
    We must verify that
    \[\begin{tikzcd}[ampersand replacement=\&]
    	M(X) \& {M(M(X))} \\
    	\& M(X)
    	\arrow["{\eta_{M(X)}}", from=1-1, to=1-2]
    	\arrow["\mu_X", from=1-2, to=2-2]
    	\arrow["\id_{M_X}"', from=1-1, to=2-2]
    \end{tikzcd}\]
    and
    \[\begin{tikzcd}[ampersand replacement=\&]
    	{M(M(M(X)))} \& {M(M(X))} \\
    	{M(M(X))} \& M(X)
    	\arrow["{\mu_{M(X)}}", from=1-1, to=1-2]
    	\arrow["\mu_X", from=1-2, to=2-2]
    	\arrow["{M(\mu_X)}"', from=1-1, to=2-1]
    	\arrow["\mu_X"', from=2-1, to=2-2]
    \end{tikzcd}\]
    commute. But the first diagram is the right-hand square of \ref{def:monad:condition:id} in Definition~\ref{def:monad}, and the second diagram is exactly \ref{def:monad:condition:assoc} in Definition~\ref{def:monad}.
\end{proof}

We summarize these examples of monads and their algebras in the following table:

\begin{table}[h]
    \centering
    \def\arraystretch{1.5}
    \begin{tabular}{c|c|c}
         $\C$ & $\M$ & $\M$-algebras \\ \hline
         $\mathsf{Set}$ & formal $\ZZ$-linear combinations & abelian groups \\
         $\mathsf{Ab}$ & \makecell{$R \otimes {-}$ \\ {\color{gray} ($R \in \mathsf{Ring}$)}}  & left $R$-modules \\
         $\mathsf{Set}$ & power set & complete semilattices \\
         $\mathsf{Set}$ & finite strings & monoids \\
         $\mathsf{Set}$ & finite probability measures & abstract convex spaces \\
         $\mathsf{Top}$ & free basepoint & pointed spaces \\
         \makecell{$\mathcal{P}(X)$ \\ {\color{gray} ($X \in \mathsf{Top}$)}} & closure & closed subsets of $X$ \\
         $\mathsf{Quiv}$ & paths & small categories
    \end{tabular}
    \caption{Examples of categories $\C$, monads $\M$ on $\C$, and characterizations of $\M$-algebras.}
    \label{table:examples:monads-and-algebras}
\end{table}

For each of these examples, there is a familiar category of $\M$-algebras. For example, a morphism of pointed spaces $(X,x) \to (Y,y)$ is a morphism $f : X \to Y$ such that $f(x) = y$. Indeed, for any monad $\M$, there is category $\M{-}\Alg$ of $\M$-algebras, defined as follows:

\begin{defn}
    Let $\C$ be a category, and let $\M = (M,\mu,\eta)$ be a monad on $\C$. The category $\M{-}\Alg$ of $\M$-algebras has:
    \begin{itemize}
        \item Objects $\M$-algebras;
        \item Morphisms $(X,e) \to (Y,f)$ those morphisms $\varphi : X \to Y$ such that
        \[\begin{tikzcd}
        M(X) \rar["M(\varphi)"] \dar["e"] & M(Y) \dar["f"] \\
        X \rar["\varphi"] & Y
        \end{tikzcd}\]
        commutes;
        \item Composition inherited from $\C$.
    \end{itemize}
\end{defn}

We can now enhance our previous proposition about free $\M$-algebras: the free construction is (as you might expect) left adjoint to a forgetful functor.

\begin{prop}\label{prop:EM-adjunction}
Let $\M = (M,\mu,\eta)$ be a monad on a category $\C$. There is an adjunction
% https://q.uiver.app/?q=WzAsMixbMCwwLCJcXG1hdGhiZntNfXstfVxcbWF0aHNme2FsZ30iXSxbMCwxLCJcXEMiXSxbMCwxLCJVX3tcXG1hdGhiZntNfX0iLDAseyJvZmZzZXQiOi0yfV0sWzEsMCwiRl97XFxtYXRoYmZ7TX19IiwwLHsib2Zmc2V0IjotMn1dXQ==
\[\begin{tikzcd}[ampersand replacement=\&]
	{\M{-}\Alg} \\
	\C
	\arrow["{U_{\M}}", shift left=2, from=1-1, to=2-1]
	\arrow["{F_{\M}}", shift left=2, from=2-1, to=1-1]
\end{tikzcd}\]
where $U_{\M} : \M{-}\Alg \to \C$ is the forgetful functor $(X,e) \mapsto X$ and $F_{\M} : \C \to \M{-}\Alg$ is the free functor $X \mapsto (M(x), \mu_X)$.
\end{prop}
\begin{proof}
    It is left as an exercise to state and check the functoriality of $U_{\M}$ and $F_{\M}$. We will verify the adjunction.
    
    So, let $X \in \C$ and $(Y,f) \in \M{-}\Alg$ be arbitrary. Given any morphism $t : X \to Y$ in $\C$, we claim that the composition $M(X) \xrightarrow{M(t)} M(Y) \xrightarrow{f} Y$ is a morphism $F_{\M}(X) = (M(X),\mu_X) \to (Y,f)$ in $\M{-}\Alg$. For this, we must check that
    % https://q.uiver.app/?q=WzAsNCxbMCwwLCJNKE0oWCkpIl0sWzEsMCwiTShZKSJdLFswLDEsIk0oWCkiXSxbMSwxLCJZIl0sWzAsMSwiTShmIFxcY2lyYyBNKFxcdmFycGhpKSkiXSxbMCwyLCJcXG11X1giLDJdLFsyLDMsImYgXFxjaXJjIE0oXFx2YXJwaGkpIiwyXSxbMSwzLCJmIl1d
    \[\begin{tikzcd}[ampersand replacement=\&]
    	{M(M(X))} \& {M(Y)} \\
    	{M(X)} \& Y
    	\arrow["{M(f \circ M(t))}", from=1-1, to=1-2]
    	\arrow["{\mu_X}"', from=1-1, to=2-1]
    	\arrow["{f \circ M(t)}"', from=2-1, to=2-2]
    	\arrow["f", from=1-2, to=2-2]
    \end{tikzcd}\]
    commutes. This square can be decomposed into two:
    % https://q.uiver.app/?q=WzAsNixbMCwwLCJNKE0oWCkpIl0sWzAsMSwiTShYKSJdLFsxLDAsIk0oTShZKSkiXSxbMSwxLCJNKFkpIl0sWzIsMCwiTShZKSJdLFsyLDEsIlkiXSxbMCwxLCJcXG11X1giLDJdLFswLDIsIk0oTShcXHZhcnBoaSkpIl0sWzEsMywiTShcXHZhcnBoaSkiLDJdLFsyLDMsIlxcbXVfWSJdLFsyLDQsIk0oZikiXSxbNCw1LCJmIl0sWzMsNSwiZiIsMl1d
    \[\begin{tikzcd}[ampersand replacement=\&]
    	{M(M(X))} \& {M(M(Y))} \& {M(Y)} \\
    	{M(X)} \& {M(Y)} \& Y
    	\arrow["{\mu_X}"', from=1-1, to=2-1]
    	\arrow["{M(M(t))}", from=1-1, to=1-2]
    	\arrow["{M(t)}"', from=2-1, to=2-2]
    	\arrow["{\mu_Y}", from=1-2, to=2-2]
    	\arrow["{M(f)}", from=1-2, to=1-3]
    	\arrow["f", from=1-3, to=2-3]
    	\arrow["f"', from=2-2, to=2-3]
    \end{tikzcd}\]
    The left-hand square commutes by naturality of $\mu$, and the right-hand square commutes because $(Y,f)$ is an $\M$-algebra. So, $f \circ M(t)$ is a morphism $F_{\M}(X) \to (Y,f)$ in $\M{-}\Alg$. This defines a function
    \[\alpha_{X,(Y,f)} : \Hom_{\C}(X,U_{\M}(Y,f)) \to \Hom_{\M{-}\Alg}(F_{\M}(X),(Y,f)).\]
    There is also a function
    \[\beta_{X,(Y,f)} : \Hom_{\M{-}\Alg}(F_{\M}(X),(Y,f)) \to \Hom_{\C}(X,U_{\M}(Y,f))\]
    given by $\varphi \mapsto \varphi \circ \eta_X$.
    Now, given any morphism $\varphi : F_{\M}(X) \to (Y,f)$ in $\M{-}\Alg$, we consider the diagram
    % https://q.uiver.app/?q=WzAsNSxbMCwwLCJNKFgpIl0sWzEsMCwiTShNKFgpKSJdLFsyLDAsIk0oWSkiXSxbMiwxLCJZIl0sWzEsMSwiTShYKSJdLFswLDEsIk0oXFxldGFfWCkiXSxbMSwyLCJNKFxcdmFycGhpKSJdLFsyLDMsImYiXSxbMSw0LCJcXG11X1giXSxbMCw0LCJcXGlkX3tNKFgpfSIsMl0sWzQsMywiXFx2YXJwaGkiLDJdXQ==
    \[\begin{tikzcd}[ampersand replacement=\&]
    	{M(X)} \& {M(M(X))} \& {M(Y)} \\
    	\& {M(X)} \& Y
    	\arrow["{M(\eta_X)}", from=1-1, to=1-2]
    	\arrow["{M(\varphi)}", from=1-2, to=1-3]
    	\arrow["f", from=1-3, to=2-3]
    	\arrow["{\mu_X}", from=1-2, to=2-2]
    	\arrow["{\id_{M(X)}}"', from=1-1, to=2-2]
    	\arrow["\varphi"', from=2-2, to=2-3]
    \end{tikzcd}\]
    The left-hand triangle commutes because $\M$ is a monad, and the right-hand square commutes because $\varphi : F_{\M}(X) \to (Y,f)$ is an $\M$-algebra homomorphism. Thus,
    \[\alpha_{X,(Y,f)}(\beta_{X,(Y,f)}(\varphi)) = f \circ M(\varphi \circ \eta_X) = f \circ M(\varphi) \circ M(\eta_X) = \varphi.\]
    Likewise, for any $t \in \Hom_{\C}(X,U_{\M}(Y,f))$, we consider the diagram
    % https://q.uiver.app/?q=WzAsNSxbMCwwLCJYIl0sWzEsMCwiWSJdLFsyLDEsIlkiXSxbMCwxLCJNKFgpIl0sWzEsMSwiTShZKSJdLFswLDEsInQiXSxbMSwyLCJcXGlkX1kiXSxbMCwzLCJcXGV0YV9YIiwyXSxbMyw0LCJNKHQpIiwyXSxbMSw0LCJcXGV0YV9ZIiwyXSxbNCwyLCJmIiwyXV0=
    \[\begin{tikzcd}[ampersand replacement=\&]
    	X \& Y \\
    	{M(X)} \& {M(Y)} \& Y
    	\arrow["t", from=1-1, to=1-2]
    	\arrow["{\id_Y}", from=1-2, to=2-3]
    	\arrow["{\eta_X}"', from=1-1, to=2-1]
    	\arrow["{M(t)}"', from=2-1, to=2-2]
    	\arrow["{\eta_Y}"', from=1-2, to=2-2]
    	\arrow["f"', from=2-2, to=2-3]
    \end{tikzcd}\]
    The left-hand square commutes by naturality of $\eta$, and the right-hand triangle commutes because $(Y,f)$ is an $\M$-algebra. Thus,
    \[\beta_{X,(Y,f)}(\alpha_{X,(Y,f)}(t)) = f \circ M(t) \circ \eta_X = f.\]
    Since $t$ and $\varphi$ are arbitrary, we conclude that $\alpha_{X,(Y,f)}$ is an isomorphism. Finally, we must check that the transformation $\alpha = (\alpha_{X,(Y,f)})_{X,(Y,f)}$ is natural in both $X$ and $(Y,f)$. So, let $h : X' \to X$ and $\psi : (Y,f) \to (Y',f')$ be arbitrary morphisms. Then we wish to show that
    % https://q.uiver.app/?q=WzAsNCxbMCwwLCJcXEhvbV97XFxDfShYLFVfe1xcbWF0aGJme019fShZLGYpKSJdLFsyLDAsIlxcSG9tX3tcXG1hdGhiZntNfXstfVxcQWxnfShGX3tcXG1hdGhiZntNfX0oWCksKFksZikpIl0sWzAsMiwiXFxIb21fe1xcQ30oWCcsVV97XFxtYXRoYmZ7TX19KFknLGYnKSkiXSxbMiwyLCJcXEhvbV97XFxtYXRoYmZ7TX17LX1cXEFsZ30oRl97XFxtYXRoYmZ7TX19KFgnKSwoWScsZicpKSJdLFswLDEsIlxcYWxwaGFfe1gsKFksZil9Il0sWzAsMiwiKGgsVV97XFxtYXRoYmZ7TX19KFxccHNpKSlfKiIsMl0sWzIsMywiXFxhbHBoYV97WCcsKFknLGYnKSl9IiwyXSxbMSwzLCIoRl97XFxtYXRoYmZ7TX19KGgpLFxccHNpKV8qIl1d
    \[\begin{tikzcd}[ampersand replacement=\&]
    	{\Hom_{\C}(X,U_{\M}(Y,f))} \&\& {\Hom_{\M{-}\Alg}(F_{\M}(X),(Y,f))} \\
    	\\
    	{\Hom_{\C}(X',U_{\M}(Y',f'))} \&\& {\Hom_{\M{-}\Alg}(F_{\M}(X'),(Y',f'))}
    	\arrow["{\alpha_{X,(Y,f)}}", from=1-1, to=1-3]
    	\arrow["{(h,U_{\M}(\psi))_*}"', from=1-1, to=3-1]
    	\arrow["{\alpha_{X',(Y',f'))}}"', from=3-1, to=3-3]
    	\arrow["{(F_{\M}(h),\psi)_*}", from=1-3, to=3-3]
    \end{tikzcd}\]
    commutes. So, let $t \in {\Hom_{\C}(X,U_{\M}(Y,f))}$ be arbitrary. Then
    \[(F_{\M}(h),\psi)_*(\alpha_{X,(Y,f)}(t)) = \psi \circ (f \circ M(t)) \circ M(h) = \psi \circ f \circ M(t \circ h)\]
    and
    \[\alpha_{X',(Y',f'))}((h,U_{\M}(\psi))_*(t)) = f' \circ M(\psi \circ t \circ h) = f \circ M(\psi) \circ M(t \circ h).\]
    Since $\psi : (Y,f) \to (Y',f')$ is a morphism in $\M{-}\Alg$, we have $\psi \circ f = f' \circ M(\psi)$, so indeed this square commutes.
\end{proof}

This proposition shows that every monad gives rise to an adjunction. In fact, every monad gives rise to \emph{two} natural adjunctions, and in the other direction every adjunction gives rise to a monad! In the next section, we'll investigate this phenomenon.

\section{Adjunctions}

Let $F \dashv G$ be any adjunction between categories $\C$ and $\D$ (that is, $F : \C \to \D$ is left adjoint to $G : \D \to \C$), defined by a unit $\eta : \id_{\C} \to G \circ F$ and counit $\varepsilon : F \circ G \to \id_{\D}$. Then $G \circ F : \C \to \C$ is an endofunctor of $\C$ with a natural monad structure, whose unit map is simply $\eta$ and whose multiplication map is the whiskering
\[G \circ F \circ G \circ F \xrightarrow{G \circ \varepsilon \circ F} G \circ \id_{\D} \circ F = G \circ F.\]
In other words, $(G \circ F, G \circ \varepsilon \circ F, \eta)$ is a monad on $\C$. So, every adjunction gives rise to a monad. However, this correspondence is not bijective -- many different adjunctions may yield the same monad! For a fixed monad $\M$, the collection of adjunctions which realize $\M$ has some structure, which we'll now study.

\begin{defn}
Let $\M$ be a monad on a category $\C$. An \emph{$\M$-adjunction} is an adjunction
% https://q.uiver.app/?q=WzAsMixbMCwwLCJcXEQiXSxbMCwxLCJcXEMiXSxbMCwxLCJHIiwwLHsib2Zmc2V0IjotMn1dLFsxLDAsIkYiLDAseyJvZmZzZXQiOi0yfV0sWzMsMiwiIiwwLHsibGV2ZWwiOjEsInN0eWxlIjp7Im5hbWUiOiJhZGp1bmN0aW9uIn19XV0=
\[\begin{tikzcd}[ampersand replacement=\&]
	\D \\
	\C
	\arrow[""{name=0, anchor=center, inner sep=0}, "G", shift left=2, from=1-1, to=2-1]
	\arrow[""{name=1, anchor=center, inner sep=0}, "F", shift left=2, from=2-1, to=1-1]
	\arrow["\dashv"{anchor=center}, draw=none, from=1, to=0]
\end{tikzcd}\]
such that the monad $(G \circ F, G \circ \varepsilon \circ F, \eta)$ equals $\M$. We say that $\D$ is the \emph{target category} of the adjunction.
\end{defn}

The above definition is of course evil, but don't worry about that for now. Let's consider our favorite example: let $\M$ be the ``formal $\ZZ$-linear combinations'' monad on $\mathsf{Set}$, so that $\M{-}\Alg$ is (equivalent to) the category of abelian groups. There is one $\M$-adjunction that we've already seen, namely $F_{\M} \dashv U_{\M}$ from Proposition~\ref{prop:EM-adjunction}, with target category $\M{-}\Alg$. However, we can notice that the image of $F_{\M}$ lies in the full subcategory of torsion-free abelian groups, and so (co)restricting $F_{\M}$ and $U_{\M}$ to this subcategory does not change the composite $U_{\M} \circ F_{\M}$. Thus, (co)restricting these functors gives a different $\M$-adjunction, with the target category being the category of torsion-free abelian groups. Of course, we can (co)restrict further to the category of free abelian groups to yield yet another $\M$-adjunction! And, even smaller, we can restrict to the image of $F_{\M}$, which consists only of those abelian groups which are equal to the free abelian group on some set.

This is the general story: for any monad $\M$, there is a ``largest possible'' target category for $\M$-adjunctions (namely $\M{-}\Alg$) and a ``smallest possible'' target category (``formally free $\M$-algebras''). Every target category of an $\M$-adjunction will sit between these two in a canonical way.

\begin{defn}
Let $\M$ be a monad on a category $\C$. $\M{-}\Free$ denotes the image of $F_{\M}$ in $\M{-}\Alg$, i.e. the full subcategory of $\M{-}\Alg$ spanned by the objects $F_{\M}(X)$ for all $X \in \C$.
\end{defn}

\begin{prop}\label{prop:comparison}
Let $\M$ be a monad on a category $\C$, and let $F \dashv G$ be an $\M$-adjunction with target category $\D$. There are unique functors $I : \M{-}\Free \to D$ and $J : \D \to \M{-}\Alg$ such that the following diagrams commute:
% https://q.uiver.app/?q=WzAsNCxbMSwxLCJcXEMiXSxbMSwwLCJcXEQiXSxbMCwwLCJcXE17LX1cXEZyZWUiXSxbMiwwLCJcXE17LX1cXEFsZyJdLFswLDIsIkZfe1xcTX0iXSxbMCwxLCJGIiwyXSxbMCwzLCJGX3tcXE19IiwyXSxbMiwxLCJJIiwwLHsib2Zmc2V0IjoxLCJzdHlsZSI6eyJib2R5Ijp7Im5hbWUiOiJkYXNoZWQifX19XSxbMSwzLCJKIiwwLHsib2Zmc2V0IjoxLCJzdHlsZSI6eyJib2R5Ijp7Im5hbWUiOiJkYXNoZWQifX19XV0=
\[\begin{tikzcd}[ampersand replacement=\&]
    {\M{-}\Free} \& \D \& {\M{-}\Alg} \\
    \& \C
    \arrow["{F_{\M}}", from=2-2, to=1-1]
    \arrow["F"', from=2-2, to=1-2]
    \arrow["{F_{\M}}"', from=2-2, to=1-3]
    \arrow["I", shift right=1, dashed, from=1-1, to=1-2]
    \arrow["J", shift right=1, dashed, from=1-2, to=1-3]
\end{tikzcd}\]
% https://q.uiver.app/?q=WzAsNCxbMSwxLCJcXEMiXSxbMSwwLCJcXEQiXSxbMCwwLCJcXE17LX1cXEZyZWUiXSxbMiwwLCJcXE17LX1cXEFsZyJdLFsyLDEsIkkiLDAseyJvZmZzZXQiOjEsInN0eWxlIjp7ImJvZHkiOnsibmFtZSI6ImRhc2hlZCJ9fX1dLFsxLDMsIkoiLDAseyJvZmZzZXQiOjEsInN0eWxlIjp7ImJvZHkiOnsibmFtZSI6ImRhc2hlZCJ9fX1dLFsyLDAsIlVfe1xcTX0iLDIseyJvZmZzZXQiOi0xfV0sWzEsMCwiRyIsMl0sWzMsMCwiVV97XFxNfSJdXQ==
\[\begin{tikzcd}[ampersand replacement=\&]
    {\M{-}\Free} \& \D \& {\M{-}\Alg} \\
    \& \C
    \arrow["I", shift right=1, dashed, from=1-1, to=1-2]
    \arrow["J", shift right=1, dashed, from=1-2, to=1-3]
    \arrow["{U_{\M}}"', shift left=1, from=1-1, to=2-2]
    \arrow["G"', from=1-2, to=2-2]
    \arrow["{U_{\M}}", from=1-3, to=2-2]
\end{tikzcd}\]
Moreover, $I$ and $J$ are fully faithful.
\end{prop}
We include the proof for completeness.
\begin{proof}
    Let $\eta$ and $\varepsilon$ denote the unit and counit (respectively) of $F \dashv G$ (and note that $\eta$ is also the unit of $F_{\M} \dashv U_{\M}$ because $F \dashv G$ is an $\M$-adjunction).
    
    First we will construct $I$. Consider some arbitrary object $F_{\M}(X)$ of $\M{-}\Free$. Since we want $I \circ F_{\M} = F$, we must have $I(F_{\M}(X)) = F(X)$. This determines the action of $I$ on objects. To extend this to a functor, we must define a function
    \[\Hom_{\M{-}\Free}(F_{\M}(X),F_{\M}(Y)) \to \Hom_{\D}(F(X),F(Y))\]
    for all objects $X,Y \in \C$. In fact, there is a natural bijection between these hom-sets that we can use:
    \begin{align*}
        \Hom_{\M{-}\Free}(F_{\M}(X),F_{\M}(Y)) &\cong \Hom_{\C}(X,U_{\M}(F_{\M}(Y))) \\
        &= \Hom_{\C}(X,G(F(Y))) \\
        &\cong \Hom_{\D}(F(X),F(Y))
    \end{align*}
    Explicitly, this function is
    \[\varphi \mapsto \varepsilon_{F(Y)} \circ F(U_{\M}(\varphi) \circ \eta_X).\]
    To check that $I$ is a functor, we consider an arbitrary pair of composable morphisms $\varphi : F_{\M}(X) \to F_{\M}(Y)$ and $\psi : F_{\M}(Y) \to F_{\M}(Z)$ in $\M{-}\Free$. Then
    \[I(\psi) \circ I(\varphi) = \varepsilon_{F(Z)} \circ F(U_{\M}(\psi) \circ \eta_Y) \circ I(\varphi) = \varepsilon_{F(Z)} \circ F(U_{\M}(\psi)) \circ F(\eta_Y) \circ I(\varphi).\]
    Now since $I(\varphi)$ is the mate of $U_{\M}(\varphi) \circ \eta_X : X \to G(F(Y))$, we have $F(\eta_Y) \circ I(\varphi) = F(U_{\M}(\varphi) \circ \eta_X)$. Thus,
    \begin{multline*}
        I(\psi) \circ I(\varphi)
        = \varepsilon_{F(Z)} \circ F(U_{\M}(\psi)) \circ F(U_{\M}(\varphi) \circ \eta_X)\\
        = \varepsilon_{F(Z)} \circ F(U_{\M}(\psi \circ \varphi) \circ \eta_X)
        = I(\psi \circ \varphi),
    \end{multline*}
    as desired. We conclude that $I$ is indeed a functor. Since
    \[I : \Hom_{\M{-}\Free}(F_{\M}(X),F_{\M}(Y)) \to \Hom_{\D}(F(X),F(Y))\]
    is a bijection, $I$ is fully faithful. Next, for any morphism $f : X \to Y$ in $\C$, we have 
    \begin{multline*}
        I(F_{\M}(f))
        = \varepsilon_{F(Y)} \circ F(U_{\M}(F_{\M}(f)) \circ \eta_X)
        = \varepsilon_{F(Y)} \circ F(\eta_Y \circ f)\\
        = \varepsilon_{F(Y)} \circ F(\eta_Y) \circ F(f)
        = F(f).
    \end{multline*}
    This verifies that $I \circ F_{\M} = F$. Next, we check that $G \circ I = U_{\M}$. On objects, we note that $G(I(F_{\M}(X)) = G(F(X)) = U_{\M}(F_{\M}(X))$. Next, let $\varphi : F_{\M}(X) \to F_{\M}(Y)$ be an arbitrary morphism in $\M{-}\Free$. By definition of $\M$-algebra homomorphism and the fact that $F \dashv G$ is an $\M$-adjunction, the following diagram typechecks and commutes:
    % https://q.uiver.app/?q=WzAsNCxbMCwwLCJHRkdGWCJdLFsyLDAsIkdGR0ZZIl0sWzAsMiwiR0ZYIl0sWzIsMiwiR0ZZIl0sWzAsMSwiR0ZVX3tcXE19XFx2YXJwaGkiXSxbMCwyLCJHXFx2YXJlcHNpbG9uX3tGWH0iLDJdLFsyLDMsIlVfe1xcTX1cXHZhcnBoaSIsMl0sWzEsMywiR1xcdmFyZXBzaWxvbl97Rll9Il1d
    \[\begin{tikzcd}[ampersand replacement=\&]
    	GFGFX \&\& GFGFY \\
    	\\
    	GFX \&\& GFY
    	\arrow["{GFU_{\M}\varphi}", from=1-1, to=1-3]
    	\arrow["{G\varepsilon_{FX}}"', from=1-1, to=3-1]
    	\arrow["{U_{\M}\varphi}"', from=3-1, to=3-3]
    	\arrow["{G\varepsilon_{FY}}", from=1-3, to=3-3]
    \end{tikzcd}\]
    Thus,
    \begin{multline*}
        G(I(\varphi))
        = G(\varepsilon_{F(Y)} \circ F(U_{\M}(\varphi) \circ \eta_X))
        = G(\varepsilon_{F(Y)}) \circ G(F(U_{\M}(\varphi))) \circ G(F(\eta_X))\\
        = U_{\M}(\varphi) \circ G(\varepsilon_{F(X)}) \circ G(F(\eta_X))
        = U_{\M}(\varphi) \circ G(\varepsilon_{F(X)} \circ F(\eta_X))\\
        = U_{\M}(\varphi) \circ G(\id_{F(X)})
        = U_{\M}(\varphi) \circ \id_{G(F(X))}
        = U_{\M}(\varphi).
    \end{multline*}
    We conclude that $G \circ I = U_{\M}$ as desired. Finally, we must check that $I$ is unique. So, suppose that $I' : \M{-}\Free \to \D$ is a functor such that $I' \circ F_{\M} = F$ and $G \circ I' = U_{\M}$. On objects, we must have $I'(F_{\M}(X)) = F(X)$, so it now remains to check that $I$ and $I'$ agree on morphisms.
    {\color{red}[TODO] FINISH}
\end{proof}

\section{The Monadicity Theorem}

Since monadic functors have nice properties, it is helpful to be able to recognize them when they appear. In other words, given a right adjoint $U : \D \to \C$ inducing a monad $\mathbf{M}$ on $\C$, we wish to know when the comparison functor $\D \to \mathbf{M}{-}\Alg$ from Proposition~\ref{prop:comparison} is an equivalence. The famous \emph{Beck Monadicity Theorem} (sometimes called the ``Barr-Beck Monadicity Theorem'', although it is not due to Barr) fully answers this question.

\begin{thm}[Beck]
A functor $U : \D \to \C$ is monadic if and only if
\begin{enumerate}
\item $U$ has a left adjoint;
\item $U$ reflects isomorphisms;
\item $\D$ has equalizers of all $U$-split pairs, and $U$ preserves these equalizers.
\end{enumerate}
\end{thm}

The first condition is of course necessary, since monadic functors must be right adjoints by definition. The second condition is also necessary, as we saw in the previous section. The third condition is somewhat more opaque, so we will take a moment to spell it out here.

\begin{defn}
    Let $U : \D \to \C$ be a functor. A pair $(f,g)$ of parallel morphisms $X \to Y$ in $\D$ is said to be \emph{$U$-split} if the pair $(Uf,Ug)$ has a split coequalizer.
\end{defn}

Now again we have an unfamiliar notion: what is a split coequalizer?

\begin{defn}
    A pair $(f,g)$ of parallel morphisms $X \to Y$ in a category $\C$ is said to have a \emph{split coequalizer} if there exists a morphism $e : Y \to Z$, $s : Z \to Y$, and $q : Y \to X$ such that
    $e \circ f = \circ g$, $e \circ s = \id_Z$, $s \circ e = g \circ q$, and $f \circ q = \id_Y$.
\end{defn}

\section{Application}

\end{document}
